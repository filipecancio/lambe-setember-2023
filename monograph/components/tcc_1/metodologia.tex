\chapter{Metodologia}\label{ch:metodologia}

\section{Classificação da pesquisa}\label{sec:classificacao-da-pesquisa}

O projeto tem por natureza mostrar como se dá a elaboração de um aplicativo utilizando práticas já usadas no mercado através de levantamento de documentações e estudos de caso de aplicativos já usados no mercado.
Assim trata-se de uma pesquisa qualitativa e com caráter exploratório.

\section{Procedimentos metodológicos}\label{sec:procedimentos-metodologicos}

O projeto terá por etapa inicial o levantamento de requisitos para entender a necessidade do profissional da economia criativa, considerando o recorte escolhido, o artista indenpendente; e a partir desse levantamento entender as funcionalidades a serem definidas e organizar quais deverão ser colocadas ou não dentro do aplicativo.

Em seguida será realizado o desenvolvimento do protótipo do projeto, cujo será dividido em jornadas, ou fluxos mediante ao que foi proposto nos casos de uso dos requisitos.
Realizaremos também a criação de um modelo de componentes de tela para que o aplicativo tenha sua própria identidade visual e uma boa experiência de usuário.

O projeto terá uma fase de desenvolvimento onde serão utilizadas ferramentas de desenvolvimento mobile nativo em Android.
O GitHub também estará presente como controle de mudanças do aplicativo e controle de versão em conjunto com o Git.

Por fim, faremos testes unitários e de integração para controle e qualidade do projeto, sendo assim viável sua disponibilização na Play Store.
