\chapter{Introdução}\label{ch:introducao}
\section{Introdução}

Há pouco mais de uma década, a sociedade revolucionou suas atividades com o uso de smartphones.
Um dispositivo que até então tinha sua principal atividade em se comunicar por voz e texto, passou a ser ferramenta de trabalho nas diversas áreas profissionais.
Por ser mais portátil e de equivalência funcional a outros aparelhos, como computadores e notebooks, o uso de aparelhos mobile não apenas cresceu, como também passou colocar os demais em segundo plano para determinadas áreas.
Um exemplo disso são as transações bancárias que passaram a ser 67,1\% online em 2022~\cite{bebraban2022}.
Esse aumento de aplicativos e usuários tem oferecido à desenvolvedores diversas oportunidades de trabalho para desenvolvimento de aplicativos de celular, visto que é uma necessidade das empresas se adequarem a essa realidade.

A economia criativa vem crescendo nos últimos anos e junto com ela os desafios na gestão do seu negócio.
De acordo com \citeonline{bittencourt}, os celulares são ferramenta importante para a economia criativa, em especial para a geração Y que utiliza de aplicativos para engajamento social e gestão.
Com isso a proposta de um aplicativo que serve de ferramenta para profissionais de economia criativa, a gerir o negócio, divulgar os trabalhos, e controlar renda.

Considerando a importância de aplicativos mobile para um público que busca cada vez mais soluções modernas para a gestão de economia criativa, veremos neste projeto de pesquisa o desenvolvimento de um aplicativo mobile que permite o gerenciamento do negócio de pessoas e empresas que contribuem para a economia criativa.

\section{Justificativa}\label{sec:justificativa}

Em 2023 chegamos a 5,44 bilhões de pessoas utilizando dispositivos móveis para uso cotidiano, lazer, atividades físicas e afins~\cite{wearesocial}, quase 70\% da população mundial.
Dentre essas atividades podemos ver destaque as transações online de bancos digitais, que só aqui no Brasil atingiu em 56\% por aplicativos de celulares em 2020 segundo a~\citeonline{bebraban2022}.
Podemos também evidenciar o uso crescente das redes sociais em dispositivos móveis, que no mundo já possui cerca de 60\% de usuários ativos~\cite{wearesocial} e que assim como os bancos digitais, tem investido cada vez mais em funcionalidades de transações e vendas online.

Em uma das áreas que alavancaram com aumento dos dispositivos, está a economia criativa que aqui no Brasil tem crescido bastante nas atividades relacionadas à área de tecnologia.
É o que consta o Mapeamento da Indústria Criativa de 2022 realizado pela~\citeonline{firjam}.
Nele podemos observar que houve uma alta de 11,7\% na geração de empregos na indústria criativa entre 2017 e 2020, cerca de 935 mil pessoas e o crescimento no setor da tecnologia da economia criativa nesses anos foi de 12,8\%.
Ainda no mesmo relatório observando a área de Tecnologia, que junto com a área de Consumo correspondem a 85\% dos empregos formais da economia criativa e ajudaram no saldo positivo desses últimos anos, podemos entender o surgimento de novas funcionalidades nas redes sociais, plataformas online e serviços de streaming que permitem aos criadores de conteúdo criativo a tornar o seu portifólio mais rentável.

Na área da economia criativa é necessário talento, prática e dedicação, coisas que exigem tempo e bastante estudo.
Um dos desafios é tornar seus produtos forma de renda, onde as atividades passam a se tornar tarefas, seu portfólio em entregáveis, e a sua rotina de pesquisador em uma rotina de empreendedor.
Apesar disso, a economia criativa possui uma parcela relevante no mercado financeiro do país e possui uma relação positiva com o aumento do uso de celulares e outros dispositivos eletrônicos~\citeonline{bittencourt}.
A utilização de dispositivos portáteis e redes sociais estreitaram as iterações sociais e formas de trabalho.

Com isso surge a proposta de um aplicativo onde o profissional que faz parte da economia criativa possa gerir o negócio, divulgar os trabalhos e controlar renda.

Em homenagem aos populares cartazes colados em ruas e postes por artistas desde o século passado, vulgo ¨lambe-lambe¨, e aos fotógrafos ambulantes que eram chamados pelo mesmo termo, o aplicativo “Lambe” serve como ferramenta para quem usa de recursos criativos como forma de renda, possuindo controle de clientes, produtos, matéria prima e venda, além de um portfólio online integrado a redes sociais e gateways de pagamento.

\section{Objetivos}\label{sec:objetivos}

\subsection{Geral}\label{subsec:geral}

O objetivo desse projeto é desenvolver um aplicativo que sirva de ferramenta de gestão para a contribuintes da economia criativa em sua jornada profissional.

\subsection{Específicos}\label{subsec:especificos}

\begin{enumerate}
    \item Realizar o levantamento dos requisitos do aplicativo baseando em informações já disponíveis em sistemas para gestão empresarial;
    \item Desenhar o protótipo e a arquitetura do aplicativo;
    \item Desenvolver o aplicativo utilizando tecnologias mobile;
    \item Realizar os testes necessários para um bom funcionamento do mesmo e disponibilizá-lo na Play Store.
\end{enumerate}