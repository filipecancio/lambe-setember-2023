\chapter{Referencial Teórico}\label{ch:referencial-teorico}

\section{Economia Criativa}\label{sec:economia-criativa}

Segundo as Nações Unidas~\cite{sebrae}, o termo "economia criativa" é dado a toda forma de expressão criativa envolvendo trabalhos artísticos e intelectuais podendo ser bens tangíveis ou intangíveis onde é possível trazer um retorno rentável para os seus produtores e envolvidos.
O termo começou a ter destaque no fim da década de 90 e desde então faz parte da pauta econômica em diversos órgãos no mundo, incluindo aqui no Brasil o IBGE, a  SEBRAE e a FIRJAN que analisa e levanta dados sobre o impacto financeiro da economia criativa do país.

A economia criativa no Brasil cresce bastante ano após ano e tem se tornado cada vez mais significativa no PIB do país, cerca de R\$ 217,4 bilhões em 2022~\cite{firjam}.
Diferente de outras economias tradicionais como agricultura, manufatura e pecuária, a economia criativa se reinventa em formas diferentes mediante a criatividade humana surgindo novas modalidades a cada ano.
Ela envolve desde a venda de jogos eletrônicos, a produção de peças teatrais, arte online com NFT, tatuagem, entre outros.

Com essa variedade de possibilidades a FIRJAM em seu Mapeamento da Indústria Criativa de 2022 conceituou quatro áreas criativas: consumo, mídias, cultura e tecnologia.
Enquanto na área de Consumo podemos considerar a produção de moda, arquitetura e publicidade; a área de mídias envolve a produção de audiovisual; a área de Cultura envolvendo música, teatro, demais artes e patrimônios de teor cultural e por fim a área de Tecnologia que fortemente relacionada ao uso do meio digital como entretenimento, lazer e arte.

Ainda no mesmo relatório observando a área de Tecnologia, que junto com a área de Consumo correspondem a 85\% dos empregos formais da economia criativa e ajudaram no saldo positivo desses últimos anos~\cite{firjam}, podemos entender o surgimento de novas funcionalidades nas redes sociais, plataformas online e serviços de streaming que permitem aos criadores de conteúdo criativo a tornar o seu portifólio mais rentável.

Bons exemplos disso são as ferramentas de marketplace da Meta (Facebook,Whatsapp Business e loja do Instagram) e também as ferramentas para cantores, bandas e “podcasters” no Spotify for Artists e de engajamento social como o TikTok que não apenas divulgam os trabalhos de artistas como portifólio aberto para o mundo todo, como possibilitam premiações e participações em playlists importantes como a Billboard~\cite{tiktok}.

\section{Desenvolvimento Mobile}\label{sec:desenvolvimento-mobile}

\subsection{Android}\label{subsec:android}

O Android é um sistema operacional desenvolvido pela Google em 2008~\cite{rpt} que se destina ao uso dos variados dispositivos móveis, dentre eles telefones, relógios, televisores, tablets, entre outros.
Devido a sua versatilidade e ser de código aberto~\cite{licences_android}, é atualmente o sistema operacional mais popular do mundo sendo adotado por milhões de usuários e diversas empresas fabricantes de dispositivos móveis.

Atualmente o Android possui a licença Apache 2.0~\cite{licences_android} o que implica que cada desenvolvedor que utiliza tecnologia Android tenha liberdade de realizar mudanças e melhorias atribuindo ou não as mesmas licenças.
Para um desenvolvedor disponibilizar seus aplicativos na Play Store, a marketplace usada para obtenção de aplicativos Android, é preciso pagar uma taxa inicial e daí então ele tem a liberdade de disponibilizar aplicativos gratuitamente nas plataformas desde que estes também sejam gratuitos e atendam as Políticas de Serviços da Google Play.
Softwares com fins lucrativos precisam pagar taxa de serviço para serem disponibilizados na loja, mas não são impedidos de serem instalados por outros meios.

A versatilidade e liberdade de código existente na plataforma é o que o torna tão acessível para usuários populares como para novos desenvolvedores a realizarem seus primeiros projetos com dispositivos móveis.

\subsection{Kotlin}\label{subsec:kotlin}

Kotlin é uma linguagem de programação desenvolvida pela Jetbrains, feita originalmente para suportar programação orientada a objetos e funcional, ela foi implementada para compilar na JVM (Java Virtual Machine), ter um desempenho acima da linguagem Java e poder trabalhar em conjunto~\cite{kotlin} (projetos Java podem ter código Kotlin sem problemas de incompatibilidade).

A linguagem passou a ser utilizada para desenvolvimento Android em 2017 e em 2019 tornou-se a linguagem oficial para desenvolvimento de Android.
Possuindo uma estrutura mais moderna e prática que o Java, o kotlin é usada por mais de 60\% dos desenvolvedores Android e garante trazer um desempenho maior, com um código mais expressivo e conciso, 20\% menos propenso a falhas e com as ferramentas mais recentes e com suporte pela google~\cite{android}.

\subsection{Kotlin KTX}\label{subsec:kotlin-ktx}

Para desenvolver aplicativos Android utilizando a linguagem Kotlin é necessário um conjunto de bibliotecas e APIs que auxiliarão no controle dos censores e atuadores dos dispositivos eletrônicos, na criação de regras de negócio e manipulação de dados e informações e por fim na visualização de telas e interação com o usuário~\cite{ktx}.
O Android KTX traz consigo esse conjunto de bibliotecas que nos auxiliará no desenvolvimento do aplicativo com atividades como: A implementação de regras de negócio com a biblioteca ViewModel; a utilização de gerenciamento de estados das telas com a biblioteca LiveData; e a gestão de dados com a biblioteca do Room Database.

\subsection{Jetpack}\label{subsec:jetpack}

O Kotlin KTX faz parte de um compilado de conjuntos de bibliotecas que compõe o Android Jetpack.
Apesar de não ser um framework, o Android Jetpack guia o desenvolvedor a trabalhar em um código consistente com o mínimo de boilerplate e incorpora boas práticas de desenvolvimento no código~\cite{jetpack}.

Das ferramentas Jetpack podemos observar a Navigation que permite dividir as telas em fragmentos e manipular os fluxos entre os fragmentos de forma gráfica, a Paging 3 que permite o carregamento de informações paginadas na tela de forma rápida eficiente e com o mínimo de custo de desempenho e dados possível, e a Compose que permite a criação do projeto Android em programação reativa.