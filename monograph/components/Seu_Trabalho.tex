%%%%%%%%%%%%%%%%%%%%%%%%%%%%%%%
% Seu trabalho começa aqui
%%%%%%%%%%%%%%%%%%%%%%%%%%%%%%%
\chapter{Desenvolvimento: Seu trabalho começa aqui}

A partir daqui começa realmente o seu trabalho. Daqui para frente deverão conter as suas contribuições e o seu trabalho realizado.

\begin{CaixaVerde}
    \begin{itemize}
        \item É comum este capítulo receber o nome de Desenvolvimento, porém você é livre para dar o nome que quiser e {\bf adicionar quantos capítulos forem necessários}.
        \item Foque em seu trabalho, em suas contribuições, na parte que você mesmo fez. Logo espera-se que mais de 60\% de todo o texto do PFC seja focado em sua própria produção.
    \end{itemize}
\end{CaixaVerde}

Esta será a parte principal do texto, que contém a exposição ordenada e pormenorizada do assunto. Divide-se em seções, subseções, subsubseções, parágrafos e subparágrafos, que variam em função da abordagem do tema e do método.

O desenvolvimento deve conter detalhados todos os itens do trabalho ou relatório, com toda teoria, cálculos, tabelas criadas, tabelas de norma utilizadas, desenhos e traçados.

O desenvolvimento pode ser dividido em várias seções, conforme a sua conveniência, e são nestas seções que
todo seu trabalho se consolidará. Neste tópico é incluído a Parte Experimental, que consiste em Materiais Utilizados e Procedimentos.

No desenvolvimento deve haver um ou mais capítulos descrevendo os {\bf resultados obtidos} pelo seu trabalho. Os resultados são enriquecidos quando possuem figuras, tabelas e gráficos criados por você, fruto do seu trabalho, e se possível, compare seus resultados com outros trabalhos já existentes.

\section{Estudo de Caso}

É extremamente interessante, quando couber, um capítulo de {\bf Estudo de Caso}. Este capítulo engloba um exemplo
detalhado de aplicação de seu trabalho, indicando os pontos positivos e negativos do mesmo. É neste tópico que se defende os seus resultados, onde se
comenta os resultados encontrados em comparação com outros trabalhos relacionados, comentando os erros e
acertos.

Caso ache interessante, e seu projeto envolver prototipagem, pode-se criar um capítulo ou seção com o nome {\bf Resultados, Ensaios e Testes}.

%%%%%%%%%%%%%%%%%%%%%%%

\section{Resultados, Ensaios e Testes dos Protótipos}

{\bf Esta seção é interessante em projetos que envolva a construção de protótipos, converse com seu orientador e veja se seu projeto se enquadra.}

Aqui deve-se relatar os resultados encontrados, realizar ensaios e testes no protótipo, elaborar gráficos e tabelas com dados coletados de testes no protótipo.

Esta seção fundamenta a validação do protótipo e prova que o mesmo funciona mostrando dados, resultados, ensaios e testes realizados no mesmo. Logo, deve ser esperado testes que mostrem os limites de funcionamento do protótipo, mostrando os objetivos sendo cumpridos e mostrando qual a capacidade máxima que seu protótipo tem de realizar a tarefa na qual
ele foi projetado para cumprir, seja de velocidade, eficiência, força, tração, etc. Este capítulo é fundamental para validação do projeto-protótipo.

%%%%%%%%%%%%%%%%%%%%%%%%%%%%%

\chapter{Dicas importantes}

\section{Importância em escolher um bom Título}

A escolha do título do projeto é uma tarefa muito importante, em minha opinião pessoal, o {\bf título é a parte mais importante de um texto científico}, porque é ele que chamará a atenção dos leitores para conhecerem o seu trabalho.

O título deve chamar a atenção das pessoas para o projeto, e de preferência conter dados relevantes e detalhes inovadores e contribuições inovadoras a respeito do projeto que o tornam diferente dos demais encontrados no mercado ou nos demais projetos.

Primeiramente se você fez um trabalho, projeto, monografia ou artigo que será disponibilizado em alguma base de dados na internet, então outras pessoas no mundo lerão o seu trabalho, e de certa forma precisarão de ter facilidade em encontrá-lo.

\begin{enumerate}
    \item Imagine que seu projeto esteja em uma Base, Repositório ou Servidor na internet, logo o seu projeto é apenas mais um entre outros milhares no mundo;
    \item Imagine agora que alguém em algum lugar no mundo entre na internet com o propósito de encontrar algum artigo, trabalho ou projeto exatamente no assunto de seu trabalho, e esse alguém não lhe conhece e ainda não sabe da existência de seu projeto.
    \item Esta pessoa irá realizar buscas através de palavras chave ou pequenas frases que contemplam o assunto desejado, e encontrará milhares de coisas, entre estas milhares está o seu trabalho.
        
    \item Como esta pessoa irá selecionar dentre os milhares de links de trabalhos encontrados?
    \begin{itemize}
        \item Primeiramente ele seleciona os títulos interessantes!
        \item Aqueles trabalhos que possuírem títulos pertinentes ao que ele está buscando será baixado da internet por ele e escolhido para "dar uma olhadinha melhor".
        \item E a pessoa continuará selecionando e baixando dezenas de trabalhos para poder olhar mais tarde, e seu título for bom, o seu
        será escolhido pela pessoa, que o abrirá e lerá o RESUMO, logo o resumo é a segunda parte mais importante e deve convencer ainda mais o
        leitor, que já deu o primeiro voto de confiança em seu trabalho ao fazer o download, de que seu projeto é realmente o que ele procura e vale a pena ser lido até o fim.
    \end{itemize}
    \item Logo, seu título deve convencer a pessoa que está buscando um trabalho como o seu, de que é interessante e contém o que ele quer.
\end{enumerate}

%%%%%%%%%%%%%%%%%%%%%%%%%%%%%

\section{O que NÃO deve ser encontrado no seu trabalho}

\begin{description}
    \item [a) Informações não autorizadas por empresas:] busque as referidas autorizações;
    \item [b) Textos que não foram escritos por você] cite todas as fontes de onde o texto foi copiado;
    \item [c) Fotos, figuras, tabelas não criadas por você] cite todas as fontes onde foram copiadas:;
    \item [d) Plágio:] para não cair em plágio, cite todas as fontes, não copie nada da internet, nem de livros, revistas, etc. Lembre-se que as obras possuem Direitos Autorais, e a apropriação indevida é crime;
    \item [e) Repetição de frases:] Cuidado para não repetir frases, seja sucinto;
    \item [f) Repetição de palavras em um mesmo parágrafo:] A repetição de palavras em um mesmo parágrafo deixa o texto ruim, capriche nos sinônimos;
    \item [g) Auto-elogio ou auto-promoção:] Evite frases como: este trabalho é muito bom, é o melhor\ldots este trabalho é perfeito,\ldots é revolucionário,\ldots~ O auto-elogio empobrece o seu texto, pois o leitor pensará: será que o trabalho é tão perfeito assim?
    \item [h) Fotos, figuras, tabelas sem títulos:] colocar os títulos em todas, respeitando a posição dos mesmos: títulos de tabelas em cima, e títulos de figuras e fotos em baixo. Figuras, tabelas e fotos precisam vir com as fontes de onde foram retiradas. 
    \item [i) Figuras com resolução ruim:] Imagens com resolução ruim empobrecem seu trabalho. Dê preferência para figuras vetorizadas;     
\end{description}
    
%%%%%%%%%%%%%%%%%%%%%%%%%%%%
%\clearpage\newpage\pagebreak

\section{Erros comuns em trabalhos científicos que podem ser evitados}

Veremos alguns ``erros'' comuns encontrados em trabalhos científicos e que podem ser evitados:

{\bf PROBLEMAS COM REFERÊNCIAS E
CITAÇÕES:}

\begin{itemize}
    \item excesso de citações de documentos e textos  da internet;
    \item nunca podem conter citações no corpo do texto que não se encontram na seção de Referências, todas as citações devem obrigatoriamente serem referenciadas;
    \item nunca podem conter citações na referência que não se encontram no corpo do texto, todas as referências devem obrigatoriamente serem citadas;
    \item nunca citar páginas e blogs de internet que não tenham revisão de pessoas ou grupos científicos confiáveis;
    \item evitar citações de apostilas, slides, encontrados na internet, pois não se pode garantir a procedência e a correta revisão;  
\end{itemize}

\begin{CaixaVerde}
    Faça o possível para a citação de artigos e trabalhos com menos de 5 anos, pois nas áreas de Engenharia, a tecnologia se atualiza muito rápido, logo devem ser citados trabalhos atuais, quanto mais recentes, melhor.
\end{CaixaVerde}

{\bf PROBLEMAS COM FIGURAS, TABELAS,
GRÁFICOS E EQUAÇÕES:}

\begin{itemize}
    \item Todas as figuras, tabelas, gráficos e equações devem ser citadas no corpo do texto;
    \item Exemplo: De acordo com a figura 8, temos que\ldots
    \item Exemplo: A tabela 3 mostra um\ldots
    \item Exemplo: O gráfico da figura 2 demonstra que\ldots
    \item Exemplo: \ldots conforme mostra a equação 7.
\end{itemize}

\begin{CaixaVerde}
    Lembre-se de citar as fontes de onde foram retiradas as figuras, gráficos, tabelas e fotos, as Equações devem possuir legenda das siglas e símbolos logo abaixo das equações e também citadas no Glossário;
\end{CaixaVerde}

{\bf OUTRAS RECOMENDAÇÕES:}

\begin{itemize}
    \item Palavras de outros idiomas devem vir em \textit{itálico.}
    \item Monografias de Engenharia geralmente não são muito grandes, raramente ultrapassam 100 páginas totais, então converse com seu orientador e tente deixar o seu trabalho por volta de 70 a 80 páginas, extrapolando somente se  realmente for necessário;
    \item Foque em seu trabalho, em suas próprias contribuições, na parte que você mesmo fez, pois espera-se que mais de 60\% de todo o texto do PFC seja focado em sua própria produção.
\end{itemize}

\begin{CaixaVermelha}
   {\bf O texto NÃO deve ser relatado no futuro:} \underline{NUNCA UTILIZE:} será, serão, irá, iremos, faremos,\ldots pois dá ideia de que não está acabado, o trabalho deve relatar um projeto pronto, então deve estar no presente ou passado
\end{CaixaVermelha}

