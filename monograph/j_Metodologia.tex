\chapter{Metodologia}
\section{Metodologia}
\label{sec:Metodologia}
Este projeto consiste na criação de uma rede social dedicada à promoção da interação entre ONGs, protetores de animais e donos de pets. Seu objetivo é contribuir de forma prática para solucionar um problema social presente em Vitória da Conquista: os animais em situação de abandono. A proposta de contribuição se baseia no desenvolvimento de uma rede social específica para a temática de animais de estimação (os pets), incorporando funcionalidades que possam auxiliar nas atividades das ONGs e atender às necessidades dos donos de pets, como a divulgação de animais para adoção, encontrados, perdidos ou que necessitam de tratamento veterinário.

Com base nos objetivos deste trabalho, é possível classificar a pesquisa como uma pesquisa-ação, conforme a definição de \cite{Thiollent}.
\begin{citacao}
“É um tipo de pesquisa social com base empírica que é concebida e realizada em estreita associação com uma ação ou com a resolução de um problema coletivo e no qual os pesquisadores e os participantes representativos da situação ou do problema estão envolvidos do modo cooperativo ou participativo".\cite[p. 14]{Thiollent}.
\end{citacao}
Entende-se que o trabalho realizado até o momento segue o ciclo de ação-reflexão da pesquisa-ação. Inicialmente, foi identificado o problema do alto índice de animais de rua e planejada a criação de uma rede social para lidar com essa questão. A implementação ocorreu por meio do desenvolvimento da plataforma e espera-se que esse trabalho sirva de base para que outros possam dar continuidade e expandir o projeto. Isso envolverá o monitoramento dos resultados alcançados, reflexão sobre as ações realizadas e possíveis ajustes, se necessário. Essas etapas futuras permitirão compreender melhor o impacto do projeto e identificar formas de melhorá-lo.

Portanto, o projeto segue o ciclo da ação-reflexão da pesquisa-ação ao identificar o problema, planejar e implementar ações para lidar com a questão dos animais de rua. A continuidade do projeto permitirá avaliar e aprimorar seu impacto na redução do abandono de animais.
% Isso se deve ao fato de que o objetivo principal da pesquisa é analisar a eficácia da rede social proposta em resolver problemas de um coletivo com mudanças de procedimentos até então adotados, cujas ações implementadas durante o desenvolvimento do sistema foram avaliadas e analisadas para identificar melhorias e aprimoramentos. Assim, espera-se maximizar a eficácia nas soluções propostas, aumentando as chances de sucesso e satisfação do usuário final.



% \newpage
\section{Procedimentos Metodológicos}
\label{sec:DescricaoIntervencao}
 
O desenvolvimento deste projeto estrutura-se em duas fases distintas:
\begin{itemize}

\item \textbf{Identificação do problema e planejamento}: Nesta fase, foi realizada a observação do problema do alto índice de animais de rua em Vitória da Conquista. Foram realizadas pesquisas e análises de sistemas e ações existentes que visam lidar com essa questão, como as estratégias adotadas por ONGs e protetores de animais. Com base nas informações obtidas, foi possível definir a abordagem do projeto, que consiste na criação de uma rede social dedicada a esse tema. Durante essa definição, levaram-se em consideração as características específicas, as funcionalidades necessárias e as demandas identificadas que devem ser abordadas pela rede social. Nesta fase também foram definidas as tecnologias e frameworks utilizados.
\item \textbf{Implementação da ação}: Na fase de implementação, foi desenvolvida a rede social com base nos requisitos definidos na fase anterior. Para hospedar a plataforma, optouse-se pelos serviços de hopspedagem \gls{VPS} (Virtual Private Server) da empresa LocaWeb\footnote{LocaWeb - Empresa brasileira de hospedagem de sites, serviços de internet e computação em nuvem}. Após receber o acesso ao serviço contratado, foram realizadas as configurações do ambiente do servidor web em uma máquina com sistema operacional Linux. Isso envolveu a instalação e configuração do Apache, MariaDB, bem como a instalação de outros pacotes necessários para fornecer a plataforma. Em seguida, os arquivos do FrontEnd e BackEnd do MeuPetAqui foram importados e o acesso ao sistema foi disponibilizado. O serviço hospedado pode ser acessado em: \href{http://vps45018.publiccloud.com.br/app/}{vps45018.publiccloud.com.br}.
\end{itemize}

\newpage
\section{Organização do Trabalho}
\label{sec:OrganizacaoTrabalho}
Este Trabalho de Conclusão de Curso está organizado em 8 capítulos distribuídos na seguinte ordem: 
\begin{enumerate}
\item[\textbf{Capítulo 1}:] É discutido as motivações para a realização deste trabalho, assim como a justificativa, problema que se procura solucionar, objetivos (geral e específicos), metodologia que norteia este projeto e descrição da intervenção.


\item[\textbf{Capítulo 2}:] São apresentadas a metodologia da abordagem deste trabalho, a descrição da intervenção e a organização do trabalho.

\item[\textbf{Capítulo 3}:] São apresentados os estudos e conhecimentos necessários para elaboração deste trabalho, orientado na construção da base teórica do desenvolvimento.

\item[\textbf{Capítulo 4}:] São discutidos os trabalhos correlatos, servindo de orientação para que o sistema criado contemple abordagens já existentes e forneça soluções para questões ainda não resolvidas no segmento.

\item[\textbf{Capítulo 5}:] São apresentadas as funcionalidades e características da Rede Social MeuPetAqui.

\item[\textbf{Capítulo 6}:] São apresentados os elementos da modelagem da plataforma web criada neste projeto. os requisitos funcionais, não-funcionais, casos de uso e informações da arquitetura do sistema.
\item[\textbf{Capítulo 7}:] São apresentadas as considerações finais.
\item[\textbf{Capítulo 8}:] São apresentadas as sugestões para trabalhos futuros.
\end{enumerate}
