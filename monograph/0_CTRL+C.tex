%%%%%%%%%%%%%%%%%%%%%%%%%%%%%%%%%%%%%%%%%
%
%  Este documento é utilizado para facilitar o CTRL+C e CTRL+V das funções e comandos principais, acrescente o que vocâ achar necessário
%
%%%%%%%%%%%%%%%%%%%%%%%%%%%%%%%%%%%%%%%%%%
%%%%%%%%% FIGURA SIMPLES %%%%%%%%%%%%%%%%
% faça upload de suas figuras para a pasta arquivos
%
\begin{figure}[htb]
   \centering
   \FonteFig{\cite{Referencia} Descrição da fonte da figura}
   \includegraphics[width=Largura]{arquivos/Sua_Figura.pdf} % PDF, PNG, JPG
   \caption{Título da sua figura}
   \label{fig:Apelido-da-Figura}
\end{figure}
%
%%%%%%%%%%%%%%%%%%%%%%%%%%%%%%%%%%%%%%%%%%%%
%%%%%%%%% DUAS FIGURAS LADO A LADO %%%%%%%%
% faça upload de suas figuras para a pasta arquivos
%
\begin{figure}[htb] 
   \FonteFig{\cite{Referencia} Descrição da fonte da Figura 1}
   \begin{minipage}[b]{0.44 \linewidth}
      \fbox{\includegraphics[width=\linewidth]{arquivos/Figura1.pdf}}
      \caption{Título da figura 1}
      \label{fig:Apelido-da-Figura-1}
   \end{minipage}
   \hfill
   \begin{minipage}[b]{0.44 \linewidth}
      \fbox{\includegraphics[width=\linewidth]{arquivos/Figura2.pdf}}
      \caption{Título da figura 2}
      \label{fig:Apelido-da-Figura-2}         
   \end{minipage} 
   \FonteFig{\cite{Referencia} Descrição da fonte da Figura 2}    
\end{figure}
%
%%%%%%%%%%%%%%%%%%%%%%%%%%%%%%%%%%%%%%%%%%
%%%%%%%% TABELA SIMPLES %%%%%%%%%%%%%%%%%%
%
\begin{table}[htb]
   \centering
   \caption{Título da Tabela}
   \begin{tabular}{lrr} % l=Left(textos); r=Right(números)
      \Linha % linha grossa
      {\bf Números} & {\bf 4 Casas} & {\bf Aleatórios} \\
      \hline % linha fina      
      Número $\pi$          &    3,1416  &    125,07  \\
      Núm. de Euler $e$     &    2,7182  &     79,00  \\
      Núm. de Ouro $\phi$   &    1,6180  &  1.975,23  \\
      Raiz: $\sqrt{2}$            &    1,4142  &      8,91  \\ 
      Raiz: $\sqrt{3}$            &    1,7320  &    543,10  \\
      \Linha % linha grossa
   \end{tabular}    
   \FonteTab{\cite{Referência} Descrição da Fonte da tabela}
   \label{tab:Apelido-da-tabela}
\end{table}
%
%%%%%%%%%%%%%%%%%%%%%%%%%%%%%%%%%%%%%%%%%%%%
%%%%%%% TABELA COLORIDA %%%%%%%%%%%%%%%%%%%%
%
\begin{table}[htb]
   \centering
   \caption{Veja o alinhamento das palavras e números}
   % colore a partir da segunda linha de vermelho e verde
   \rowcolors{2}{red!20}{green!20}
   \begin{tabular}{lrlrr} % l=Left(textos); r=Right(números)
      \Linha % linha grossa
      \rowcolor{red!60} % colore a linha imediatamente abaixo
      \textbf{Fruta} & \textbf{Peso (kg)} & \textbf{Qualid.} & \textbf{R\$/kg}  &  \textbf{Total (R\$)}\\
      \Linha % linha grossa
      Maçã          & 1,250  & madura &  12,99  &  16,24 \\
      Banana Prata  & 2,510  & verde  &   8,99  &  22,56 \\
      Uva Niágara   & 1,860  & verde  &   7,09  &  13,19 \\
      Pera          & 2,110  & madura &  12,67  &  26,73 \\
      Kiwi          & 3,640  & verde  &  10,67  &  38,84 \\
      Manga Palmer  & 5,300  & madura &   5,98  &  31,69 \\
      \Linha % linha grossa
      \rowcolor{green!60} % colore a linha imediatamente abaixo
      \textbf{TOTAL} &       &        &         & {\bf 149,26} \\
      \Linha % linha grossa
   \end{tabular}    
   \FonteTab{\cite{Referência} Descrição da Fonte da Tabela}
   \label{tab:Apelido-da-Tabela}
\end{table}
%
%%%%%%%%%%%%%%%%%%%%%%%%%%%%%%%%%%%%%%%%%%%%%%%
%%%%% EQUAÇÕES SIMPLES %%%%%%%%%%%%%%%%%%%%%%%
%
\begin{equation}
   N_1 = \frac{\sqrt{2}}{2 \cdot \pi } \cdot \frac{U_1 \cdot  10^8}{f \cdot S_L \cdot B} \quad \Rightarrow \quad N_1 = \frac{U_1 \cdot 10^8}{4,44 \cdot f \cdot S_L \cdot B}
   \label{eq:Apelido-da-Equacao}
\end{equation}
%
%%% LEGENDA DA EQUAÇÃO COM TABULAÇÃO E 
%%% CITAÇÃO NO GLOSSÁRIO 
%
Sendo que:\\
\vspace{-1.5cm}
\begin{tabbing}
   \hspace{1cm}  \= \hspace{1cm} \= \kill \\
   \gls{N1}      \> $\Rightarrow$ \> Número de espiras do primário; \\
   \gls{U1}      \> $\Rightarrow$ \> Tensão do primário (V); \\
   \gls{freq}    \> $\Rightarrow$ \> frequência (Hz); \\
   \gls{SL}      \> $\Rightarrow$ \> Área da Seção Líquida do núcleo ($cm^2$); \\
   \gls{DenFlux} \> $\Rightarrow$ \> Densidade de Fluxo Magnético (G = Gauss); 
\end{tabbing}
%
%%%%%%%%%%%%%%%%%%%%%%%%%%%%%%%%%%%%%%%%%%%%%%
%%%%% EQUAÇÕES ALINHADAS EM VÁRIAS LINHAS %%%%
%
\begin{eqnarray}
   \label{eq:Apelido-da-Equacao-1}
   10x^2y+15xy^2-5xy & = & 5(2x^2y+3xy^2-xy) \\    
   \label{eq:Apelido-da-Equacao-2}
                     & = & 5x(2xy+3y^2-y) \\
   \label{eq:Apelido-da-Equacao-3}
                     & = & 5xy(2x+3y-1)    
\end{eqnarray}
%
%%%%% Para ocultar o número da equação:
%
\begin{eqnarray}
   10x^2y+15xy^2-5xy   & = & 5(2x^2y+3xy^2-xy) \nonumber \\
                       & = & 5x(2xy+3y^2-y)    \nonumber \\
   \label{eq:Apelido-da-Equacao}
                       & = & 5xy(2x+3y-1)
\end{eqnarray}
%
%%%%%%%%%%%%%%%%%%%%%%%%%%%%%%%%%%%%%%%%%%%%%%%
%%%% LEGENDAS COM TABULAÇÃO E GLOSSÁRIO %%%%%%%
%
Legenda:\\
\vspace{-1.3cm}
\begin{tabbing}
    \hspace{1cm}  \= \hspace{1cm} \= \kill \\
    \gls{Ua}      \> $\Rightarrow$ \> Tensão na Fase A ~~~(\textcolor{blue}{\rule{3cm}{1mm}}); \\
    \gls{Ia}      \> $\Rightarrow$ \> Corrente na Fase A (\textcolor{red}{\hdashrule[0.1ex]{3.1cm}{1mm}{2mm}}); \\
    \gls{fi}    \> $\Rightarrow$ \> Ângulo do Fator de Potência; \\
    \gls{w}    \> $\Rightarrow$ \> Velocidade angular: $\omega=2\cdot \pi \cdot f$;    
\end{tabbing}
%
%%%%%%%%%%%%%%%%%%%%%%%%%%%%%%%%%%%%%%%%%%%%%%%%%
%%%% CAIXA VERDE E CAIXA VERMELHA %%%%%%%%%%%%%%%
%
\begin{CaixaVerde}
    Texto para a caixa destacada em Verde.
\end{CaixaVerde}
%
\begin{CaixaVermelha}
    Texto para a caixa destacada em Vermelho.
\end{CaixaVermelha}
%
% caso queira caixas diferentes e de outras cores:
% https://www.overleaf.com/latex/examples/drawing-coloured-boxes-using-tcolorbox/pvknncpjyfbp
% 
%%%%%%%%%%%%%%%%%%%%%%%%%%%%%%%%%%%%%%%%%%%%%%%%%%%
%%%%%%% CÓDIGO - IMPORTA CÓDIGO DE ARQUIVO %%%%%%%%
% faça upload de seus códigos para a pasta arquivos
%
\ImportaCodigo[language=Nome-da-Linguagem,
        caption=Titulo do Codigo,
        label=cod:LabelCod]
        {arquivos/Arquivo-do-Seu-Codigo.cpp}
%
% o arquivo do seu código (C++, Python, etc...) 
% poderá ser colocado na pasta de arquivos 
% aqui do Template e seu código será 
% importado para o texto
%
% language= Python, C++, SQL, JAVA, ...
% linguagens suportadas, acesse:
% https://pt.overleaf.com/learn/latex/Code_listing
%
%%%%%%%%%%%%%%%%%%%%%%%%%%%%%%%%%%%%%%%%%%%%%%%%%
%%%%%% CÓDIGO DIRETAMENTE NO TEXTO %%%%%%%%%%%%%
%
\begin{Codigo}[language=Nome-Da-Linguagem, caption=Título do seu Código, label=cod:Apelido-do-seu-código]
            %
            %  Seu codigo vem aqui
            %    
\end{Codigo}
%
% language= Python, C++, SQL, JAVA, ...
% linguagens suportadas, acesse:
% https://pt.overleaf.com/learn/latex/Code_listing
%
%%%%%%%%%%%%%%%%%%%%%%%%%%%%%%%%%%%%%%%%%%%%%%%%%%%