\clearpage\newpage\pagebreak%
\pagestyle{plain}

\Resumo 

Aqui vai o texto resumo do seu trabalho. 
O {\bf Resumo} deve dar ideia do projeto como um todo, de forma que quem ler o resumo saiba reconhecer o que será detalhado no corpo do trabalho, logo {\bf o resumo deve ser completo, descrever o projeto, citar objetivos, mostrar resultados encontrados, e conclusões}, isso tudo em uma única página.
    
O Resumo deve ser a miniatura do seu projeto completo. O Resumo deve fornecer dados suficientes ao leitor para que ele entenda todo o projeto realizado, mostrar objetivos, resultados {\bf (focar bastante nos resultados encontrados)} e conclusões de forma que o leitor fique com vontade de ler o restante do trabalho, para verificar os detalhes do que acaba de ler no resumo.

O Resumo é o lugar onde você deve "Vender o Seu Trabalho", é o seu "Anúncio" que irá convencer o leitor de que seu projeto é interessante e deve ser lido e apreciado com maiores detalhes na leitura das próximas páginas.
    
O {\bf Título} do trabalho é considerado como sendo a parte mais importante do anúncio do seu trabalho, e em segundo lugar está o Resumo.

% máximo de 5 palavras chave separadas por vírgula
\begin{keywords}
Aqui, estão, as palavras, chave, importantes
\end{keywords}