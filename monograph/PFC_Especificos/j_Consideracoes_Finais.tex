\chapter{Considerações Finais}

    Em trabalhos científicos e de Engenharia, evitamos a utilização do nome {\bf Conclusão}, e adotamos o nome {\bf Considerações Finais} ao fim do trabalho, e adotamos um capítulo para fazer comentários a respeito das {\bf Sugestões para Trabalhos Futuros}.

    Neste capítulo de {\bf Considerações Finais}, deve conter uma breve comentário a respeito das etapas de realização do trabalho, desafios e dificuldades encontradas.    
    Este capítulo deve estar alicerçado nos {\bf resultados}, deve associá-los à confirmação (ou não) da(s) hipótese(s) ou pressuposto(s) se for o caso, e aos objetivos estabelecidos.

    Incluir propostas e recomendações para implementação de resultados e novas pesquisas, dando fechamento ao trabalho.
    
    Segundo a norma \cite{NBR14724_TrabalhosAcademicos} este capítulo trata-se de: Parte final do texto, na qual se apresentam conclusões correspondentes aos objetivos ou hipóteses. 

%\newpage % comando que transfere o texto para a próxima página
\section{Dicas importantes:}
    
\begin{itemize}
    \item Evite conclusões óbvias;
    \item Evite se auto promover, exaltar ou idolatrar o seu trabalho;
    \item As conclusões devem ser curtas (uma ou duas páginas) e baseadas no que foi escrito em seu trabalho;
    \item Cite os resultados encontrados e os objetivos e metas alcançadas e não alcançadas;
    \item É interessante que você recomende alguma contribuição para trabalhos futuros, que possa ajudar alguma pessoa a continuar ou repetir o seu trabalho;
    \item Aqui não devem ser colocadas tabelas nem figuras, nem equações;
    \item Aqui não deve ser colocada conclusões de outras pessoas, nem textos de outras pessoas, a não ser que sejam comparações de seus resultados com outros já sacramentados no Estado da Arte;
    \item Evitar afirmações duras e genéricas, difíceis de provar na prática;
    \item A conclusão deve possuir Clareza, Coesão e Coerência:
    \begin{itemize}
        \item {\bf Clareza:} bem explicado e de fácil entendimento;
        \item {\bf Coesão:} deve reunir e ligar todas as partes do seu trabalho;
        \item {\bf Coerência:} raciocínio sem contradições;
    \end{itemize}
    \item As Considerações Finais, juntamente com o resumo, são as seções mais importantes do texto do seu trabalho, e devem ser feitas cuidadosamente para valorizar todo o seu trabalho; 
\end{itemize}