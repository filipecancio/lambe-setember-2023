%
\chapter{Referencial Teórico}

É no Referencial Teórico que você vai indicar, citar e comentar o que outros autores já fizeram, e o que você fará no seu trabalho.

\begin{itemize}
    \item Cite os trabalhos relacionados ao tema do seu projeto; 
    \item Neste capítulo haverá citações de  outros autores, então cuidado com o plágio;
    \item Cuidado ao citar muitas referências, deve haver um equilíbrio entre o seu texto, as suas contribuições e as citações de outros autores;
\end{itemize}

É aqui que você fará comparações entre o que já se sabe no mundo a respeito do tema abordado no seu trabalho e a comparação com o seu trabalho. Você precisará mostrar o que o seu trabalho tem de semelhante e o que tem de diferente dos demais trabalhos já produzidos no mesmo tema.

No capítulo de Referencial Teórico, deve ser abordado uma breve introdução teórica a respeito do assunto do seu trabalho e um breve {\bf levantamento bibliográfico} citando e referenciando as normas a serem adotadas, manuais, artigos, livros, referências e técnicas aplicadas, e principalmente referenciar outros trabalhos semelhantes ao seu que já foram feitos por outras pessoas no Brasil e no mundo, se necessário levante o Estado da Arte e o Estado da Técnica referentes ao tema do seu trabalho. 

%%%%%%%%%%%%%%%%%%%%%%%%%%%%%
\section{Levantamento do Estado da Arte}

É no capítulo de Referencial Teórico que se levanta o \textbf{ESTADO DA ARTE}, que mostra o que já se tem estudado e o que se tem feito no tema abordado no seu trabalho:
    
\begin{description}
    \item [ESTADO DA ARTE:] O Estado da Arte é uma referência ao estado atual de conhecimento sobre um determinado tópico que está sendo objeto de análise ou estudo. De forma geral, o estado da arte representa o nível mais alto de um processo de desenvolvimento, seja de um aparelho, de uma técnica ou de uma área científica, alcançado até um determinado momento. 
\end{description}

%%%%%%%%%%%%%%%%%%%%%%%%%%%%
\section{Levantamento do Estado da Técnica}

Caso o seu trabalho venha a citar alguma {\bf patente} ou algum trabalho que esteja protegido por {propriedade intelectual}, você deve levantar o \textbf{ESTADO DA TÉCNICA}, as \textbf{INVENÇÕES} e \textbf{MODELOS DE UTILIDADE}.

\begin{description}
    \item [ESTADO DA TÉCNICA:] O Estado da técnica é um termo usado na propriedade industrial e compreende tudo aquilo que foi tornado acessível ao público antes da data de depósito do pedido de patente, por descrição escrita ou oral, por uso ou qualquer outro meio, no Brasil ou no exterior. Isso significa que a invenção e o modelo de utilidade são considerados novos somente quando não estão compreendidos no estado da técnica.
    Logo, se uma invenção ou modelo de utilidade já foi divulgado ao público antes do depósito do pedido de patente, ele é considerado parte do estado da técnica e não é mais considerado novo.
    
    \item[INVENÇÃO:] A Invenção é uma criação industrial que pode ser explorada economicamente e que atende a quatro requisitos previstos na lei de propriedade industrial (Lei nº 9.279/1996): novidade, atividade inventiva e aplicação industrial. Isso significa que para ser considerada uma invenção e ser patenteada, a criação deve ser nova, não óbvia para um técnico no assunto, ter aplicação industrial e não estar impedida por outras leis ou regulamentos.
    
    \item[MODELO DE UTILIDADE:] O Modelo de Utilidade é uma modalidade de patente que se destina a proteger inovações com menor carga inventiva, normalmente resultantes da atividade do operário ou artífice1. É o objeto de uso prático suscetível de aplicação industrial, como novo formato de que resulta melhores condições de uso ou fabricação.
\end{description}

%%%%%%%%%%%%%%%%%%%%%%%%%%%%%%

\section{Não confunda Introdução e  Referencial Teórico}

É muito frequente a confusão do que deve ser escrito no capítulo de {\bf Introdução} e o que deve ser escrito no capítulo de {\bf Referencial Teórico}.

A principal diferença é que no {\bf capítulo de Introdução} você irá mostrar ao leitor do que se trata o seu trabalho e o que o leitor irá encontrar ao ler o seu texto.

Já no {\bf capítulo de Referencial Teórico}, você irá mostrar ao leitor que você conhece o que outros autores já fizeram no mesmo tema que você está trabalhando, e que o seu trabalho é no que o seu trabalho é diferente ou complementa outros trabalhos de outros autores.

%%%%%%%%%%%%%%%%%%%%%%%%%%%%

\section{Como realizar citações automáticas de Referências}

A forma correta de se referenciar em \LaTeX é utilizando a ferramenta {\bf BibTeX}.

Uma ótima forma de utilizar o {\bf BibTeX} é criando uma biblioteca {\bf *.bib} e adicionar todas as nossas referências dentro dela.

Ver detalhes completos da utilização do {\bf BibTeX} na seção~\ref{sec:BibTeX}, página~\pageref{sec:BibTeX}.

