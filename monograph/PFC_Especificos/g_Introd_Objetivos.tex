%
\chapter{Introdução}

%%%%%%%%%%%%%%%%%%%%%%%%%%%%%%%%%

\section{Criando a Introdução do Trabalho}

A Introdução é a parte inicial do texto, onde devem constar a delimitação do assunto tratado, objetivos da pesquisa e outros elementos necessários para situar o tema do trabalho. É aqui que você vai explicar para o leitor do que se trata o seu trabalho.

De acordo com a Metodologia Científica, a Introdução é a parte do trabalho onde se {\bf apresenta o tema e o problema a serem abordados. É também onde se justifica a relevância do tema e se articulam os argumentos para demonstrar sua importância.}

Este capítulo de Introdução não pode ser muito grande, pois o foco principal deverá ser no seu próprio trabalho a ser descrito nos capítulos de Desenvolvimento e demais capítulos subsequentes. 
    
\begin{itemize}
    \item Explique ao leitor do que se trata o seu trabalho;
    \item Exponha o problema e o tema a ser abordado;
    \item Deixe claro a importância e relevância do tema abordado no seu trabalho;
    \item Se possível, responda a alguma pergunta relacionada ao problema;
    \item No fim deste capítulo é interessante que contenha um parágrafo contendo os resultados encontrados, incentivando o leitor a encontrar os detalhes destes resultados ao longo do texto do seu trabalho;
    \item Deve-se evitar figuras, tabelas, gráficos e equações na introdução pois são detalhes dos próximos capítulos do seu trabalho.
    \item Seja sucinto e limite-se ao essencial, este capítulo não precisa ser muito grande;
    \item Não fique de enrolação nem "enchendo linguiça", você não precisará de muito texto para explicar o que é o seu trabalho.
\end{itemize}

O foco do seu trabalho deve ser em suas próprias contribuições, ou seja, {\bf a parte principal de todo o trabalho é o que você desenvolveu, em seus dados, seus ensaios, testes e seus resultados encontrados}.

%%%%%%%%%%%%%%%%%%%%%%%%%%%%%%%

\section{Objetivo Geral}

O Objetivo Geral deve traduzir perfeitamente, em parágrafo único, o objetivo maior do projeto no que se diz respeito à essência da pesquisa e dos resultados esperados. O Objetivo Geral deve começar com o verbo no infinitivo: {\bf Relacionar, estudar, avaliar, desenvolver\ldots}
    
%%%%%%%%%%%%%%%%%%%%%%%%%%%%%%%

\subsection{Objetivos Específicos}

Os Objetivos Específicos devem traduzir metas parciais, que ao serem cumpridas levarão o projeto, trabalho ou assunto, a alcançar o objetivo geral proposto. 
     
Os Objetivos Específicos são consequências geradas pelo Objetivo Geral e podem vir na forma de tópicos:

\begin{enumerate}
    \item Cite um Objetivo Específico;
    \item Outro Objetivo Específico;
    \item Vamos fazendo mais objetivos específicos;
    \item Coloque quantos forem necessários;
\end{enumerate}

Durante o desenvolvimento do seu trabalho, você irá respondendo cada item citado nos Objetivos Específicos.

%%%%%%%%%%%%%%%%%%%%%%%%%%%%%%%

\section{Justificativa}

A seção de Justificativa, também pode ser chamada de Motivação. Esta seção deve deixar claro a importância do trabalho e do assunto abordado, e explorar o que motivou a realização do projeto. Esta seção não deve ser muito longa e se resume em alguns parágrafos.

    
