%
\chapter{Introdução}
\label{cap:Introdução}
\hyperlink{chap:Introdução}{}
Durante a evolução humana, foram estabelecidas relações com outras espécies, algumas das quais foram domesticadas e utilizadas como ferramentas de trabalho, parceiros de caça ou inseridos no ambiente familiar. No entanto, mesmo com as práticas desenvolvidas para o bem-estar desses animais, o abuso e o abandono ainda são comuns em nossa sociedade, apresentando uma questão social e de saúde pública.

Nas áreas urbanas, é comum que sejam encontrados cães e gatos vivendo em condições precárias, expostos a produtos químicos, objetos cortantes, doenças virais e microrganismos. A presença de animais de rua nas vias públicas pode apresentar riscos para a população, uma vez que esses animais podem transmitir vírus e parasitas e, em alguns casos, podem reagir agressivamente em defesa de seu território.

Existem diversas motivações que levam ao abandono de animais domésticos, como por exemplo a rejeição a fêmeas em processo de gestação, fuga do custeio de tratamento veterinário, idade avançada do animal, mudanças de residência, viagens, crescimento desmedido do animal, latidos ou miados excessivos, surgimento de alergias dentre outros.

De acordo com o Anuário Pet 2020 do Instituto Pet Brasil, existem mais de 172 mil animais sob cuidado de ONGs e grupos de protetores no Brasil, dos quais 96\% são cães e 4\% são gatos \cite{institutopetbrasil2020}. Esses grupos são responsáveis por manter esses animais e promover a adoção voluntária. Em uma postagem de 2020, o site Avoador informou que em Vitória da Conquista, em 2019, já existiam 6 mil animais vivendo em situação de abandono \cite{avoador2020}. Sendo este até então o único registro quantitativo encontrado a respeito da situação nesta cidade, um problema que tende a ter se agravado com a pandemia do covid-19 iniciada em 2019.

O Instituto Pet Brasil também informa que existem um total de 370 ONGs de proteção animal no Brasil, com 46\% delas localizadas na região sudeste, 18\% na região sul, 17\%, 12\% e 7\% nas regiões nordeste, norte e centro-oeste, respectivamente. Essas ONGs são classificadas como pequeno porte (até cem animais), médio porte (de cem a quinhentos animais) e grande porte (mais de quinhentos animais)\cite{institutopetbrasil2020}.	

% Modelo (Template) ~\LaTeX ~da classe: {\bf cls-PFC-COEEL.cls} criada pelo professor Elvio, para atender às demandas de \textbf{PFC - Projeto de Fim de Curso} de Engenharia Elétrica (\gls{COEEL}) do IFBA {\it campus} Vitória da Conquista e \textbf{Relatórios Técnicos} de disciplinas.

% Este modelo foi criado e testado no  {\bf \LaTeX-Overleaf} para ser compilado em {\bf pdfLaTeX} ou {\bf XeLaTeX}.

% \vspace{1cm} % espaço vertical entre as linhas
% Segue uma dica importante com o comando \verb|\begin{CaixaVermelha}|:

% \begin{CaixaVermelha}
%     {\bf Dica:} Membros do {\bf \gls{IEEE}} com membresias ativas de estudantes ou professores, possuem acesso total ao Overleaf com todas as funções desbloqueadas.
% \end{CaixaVermelha}

% \vspace{1cm} % espaço vertical entre as linhas
% Agora a mesma dica com o comando \verb|\begin{CaixaVerde}|:

% \begin{CaixaVerde}
%     {\bf Dica:} Membros do {\bf \gls{IEEE}} com membresias ativas de estudantes ou professores, possuem acesso total ao Overleaf com todas as funções desbloqueadas.
% \end{CaixaVerde}

%%%%%%%%%%%%%%%%%%%%%%%%%%%%%%%%%

\section{Motivação}
\label{sec:Motivacao}
A despeito de ser uma cidade com mais de 300 mil habitantes, Vitória da Conquista não dispõe de um Centro de Controle de Zoonoses. Como consequência, têm surgido diversos grupos de protetores e ONGs que visam auxiliar animais em situação de vulnerabilidade, promovendo a adoção, arrecadando fundos para mantê-los em locais seguros e fornecendo cuidados veterinários e demais cuidados necessários para minimizar os impactos dessa situação. De maneira geral, esses grupos utilizam perfis em redes sociais para disseminar suas campanhas e conquistar o engajamento da população, embora essas plataformas não tenham sido originalmente projetadas para atender às necessidades dessas organizações.

\section{Justificativa}
\label{sec:Justificativa}
ONGs e grupos de proteção animal atuam de diversas maneiras para diminuir a quantidade de animais abandonados, divulgando imagens dos animais perdidos, buscando os donos quando um animal é encontrado, divulgando cães e gatos disponíveis para adoção e em campanhas para arrecadação de doações para custeio de tratamentos, castrações, aquisição de rações e medicações para os animais abrigados em lares temporários. Porém, o trabalho das organizações é limitado pela capacidade de acompanhamento aos animais. A quantidade de dependentes das ONGs cresce em um ritmo que o número de adoções não consegue acompanhar. Somado a isso, muitos dos cães e gatos que são acolhidos precisam ser internados para iniciar algum tratamento de problemas relacionados à sua vivência nas ruas, culminando em momentos em que não é possível ajudar um animal em determinada situação ou é necessário acumular dívidas que dependem do apoio de simpatizantes com a causa para serem quitadas.

O trabalho dessas ONGs e grupos é mantido de forma independente em relação às instituições públicas e têm como motivação o afeto e empatia aos animais e se faz essencial para a sociedade através da contribuição direta à saúde pública. Animais abandonados sobrevivem sem cuidados básicos como alimentação adequada, higienização, vacinação, vermifugação e castração para controle populacional, neste cenário, todos fatores contribuem para que venham a adquirir doenças de origem bacteriana ou viral e que podem ser transmitidas para animais domésticos durante passeios em locais de comum acesso a estes, tornando-os possíveis vetores para a transmissão a humanos. 

\lstset{
    breaklines=true,
    postbreak=\mbox{\textcolor{red}{$\hookrightarrow$}\space}
}


Este estudo se faz relevante pelo seu potencial em contribuir com as ações desenvolvidas pelas entidades envolvidas, facilitando o processo de adoção, mantendo um histórico de adoções, ajudando a cruzar dados de animais perdidos com relatos de animais encontrados e desta forma encurtando o tempo para que o animal volte ao seu dono, reduzindo a quantidade de animais nas ruas e contribuindo diretamente para com a saúde pública e principalmente o bem-estar animal.

Ao consultar a Lei nº 9.605 de 12 de Fevereiro de 1998, o artigo 32 da mesma diz:
\begin{citacao}
ART.32. Praticar ato de abuso, maus-tratos, ferir ou mutilar animais silvestres, domésticos ou domesticados, nativos ou exóticos:

Pena - Detenção, de três meses a um ano, e multa.

§ 1º Incorre nas mesmas penas quem realiza experiência dolorosa ou cruel em animal vivo, ainda que para fins didáticos ou científicos, quando existirem recursos alternativos.

§ 1º - A Quando se tratar de cão ou gato, a pena para as condutas descritas no caput deste artigo será de reclusão, de 2 (dois) a 5 (cinco) anos, multa e proibição da guarda. (Incluído pela Lei nº 14.064, de 2020).

§ 2º A pena é aumentada de um sexto a um terço, se ocorre morte do animal. ~\cite{lei}. 
\end{citacao}

De acordo com a lei mencionada, os animais têm direitos que devem ser garantidos para protegê-los e proporcionar-lhes uma vida digna. Infelizmente, em Vitória da Conquista, é comum encontrar relatos que contradizem esses direitos, como animais abandonados ou perdidos nas ruas, adoecendo devido à subnutrição ou alimentação inadequada, sofrendo agressões de moradores ou sendo mutilados em acidentes de trânsito. Apesar disso, muitas vezes os animais são salvos graças à intervenção de ONGs e protetores de animais, que atuam de maneira independente sem o apoio de um centro de zoonoses, já que a cidade não possui essa infraestrutura.

\section{Problema}
\label{sec:Problema}
Sendo evidente e preocupante a problemática da incidência de animais abandonados em Vitória da Conquista e com base nas informações apresentadas, este trabalho visa abordar a seguinte questão: Como uma Rede Social pode ser desenvolvida para contribuir com a redução da incidência de animais em situação de abandono?

\section{Hipóteses}
Com base na questão proposta sobre como o desenvolvimento de uma  Rede Social pode contribuir para a redução da incidência de animais em situação de abandono em Vitória da Conquista, aqui estão algumas hipóteses possíveis:
\begin{itemize}
\item Hipótese: O desenvolvimento de uma Rede Social que facilite a busca por animais disponíveis para adoção pode aumentar o acesso e a visibilidade desses animais, o que pode levar a um potencial aumento nas adoções e, consequentemente, contribuir para a redução da população de animais em situação de abandono.

\item Hipótese: Uma Rede Social que promova a interação e o compartilhamento de informações sobre cuidados adequados, treinamento e dicas de comportamento para animais de estimação pode educar os proprietários e incentivar práticas responsáveis de cuidado, com o potencial de prevenir o abandono de animais.
\newpage
\item Hipótese: O desenvolvimento de uma Rede Social que permita o relato de casos de animais desaparecidos e ofereça um mecanismo de rastreamento desses animais pode agilizar as ações de resgate, aumentando as chances de reunião com seus donos e diminuindo o tempo em que os animais permanecem em situação de abandono.

\item Hipótese: Uma Rede Social que funcione como uma opção adicional de divulgação de animais assistidos por ONGs e protetores de animais pode potencialmente aumentar a conscientização sobre essas organizações, estimulando mais apoio e contribuições para a manutenção de seus serviços e, consequentemente, auxiliando na redução da população de animais em situação de abandono.
    
\end{itemize}
% A Introdução é a parte inicial do texto, onde devem constar a delimitação do assunto tratado, objetivos da pesquisa e outros elementos necessários para situar o tema do trabalho. É aqui que você vai explicar para o leitor do que se trata o seu trabalho.

% De acordo com a Metodologia Científica, a Introdução é a parte do trabalho onde se {\bf apresenta o tema e o problema a serem abordados. É também onde se justifica a relevância do tema e se articulam os argumentos para demonstrar sua importância.}

% Este capítulo de Introdução não pode ser muito grande, pois o foco principal deverá ser no seu próprio trabalho a ser descrito nos capítulos de Desenvolvimento e demais capítulos subsequentes. 
    
% \begin{itemize}
%     \item Explique ao leitor do que se trata o seu trabalho;
%     \item Exponha o problema e o tema a ser abordado;
%     \item Deixe claro a importância e relevância do tema abordado no seu trabalho;
%     \item Se possível, responda a alguma pergunta relacionada ao problema;
%     \item No fim deste capítulo é interessante que contenha um parágrafo contendo os resultados encontrados, incentivando o leitor a encontrar os detalhes destes resultados ao longo do texto do seu trabalho;
%     \item Deve-se evitar figuras, tabelas, gráficos e equações na introdução pois são detalhes dos próximos capítulos do seu trabalho.
%     \item Seja sucinto e limite-se ao essencial, este capítulo não precisa ser muito grande;
%     \item Não fique de enrolação nem "enchendo linguiça", você não precisará de muito texto para explicar o que é o seu trabalho.
% \end{itemize}

% O foco do seu trabalho deve ser em suas próprias contribuições, ou seja, {\bf a parte principal de todo o trabalho é o que você desenvolveu, em seus dados, seus ensaios, testes e seus resultados encontrados}.

%%%%%%%%%%%%%%%%%%%%%%%%%%%%%%%

\section{Objetivos}
\label{sec:Objetivos}
\subsection{Objetivo Geral}
\label{subsec:ObjetivoGeral}

Desenvolver um sistema web que auxilie nas atividades recorrentes das ONGs e protetores de animais abandonados, com o propósito de reduzir o número de cachorros e gatos em situação de abandono na cidade de Vitória da Conquista, Bahia. Além disso, o sistema fornecerá um espaço de interação entre os donos de pets, visando promover o bem-estar dos animais de estimação.

    
%%%%%%%%%%%%%%%%%%%%%%%%%%%%%%%

\subsection{Objetivos Específicos}
\label{subsec:ObjetivosEspecificos}
Aplicar técnicas de Engenharia de Software para construir uma rede social direcionada aos donos de pets e ONGs envolvidas na causa animal.
% Os Objetivos Específicos devem traduzir metas parciais, que ao serem cumpridas levarão o projeto, trabalho ou assunto, a alcançar o objetivo geral proposto. 
     
% Os Objetivos Específicos são consequências geradas pelo Objetivo Geral e podem vir na forma de tópicos:

% \begin{enumerate}
%     \item Cite um Objetivo Específico;
%     \item Outro Objetivo Específico;
%     \item Vamos fazendo mais objetivos específicos;
%     \item Coloque quantos forem necessários;
% \end{enumerate}

% Durante o desenvolvimento do seu trabalho, você irá respondendo cada item citado nos Objetivos Específicos.

%%%%%%%%%%%%%%%%%%%%%%%%%%%%%%%

% \section{Justificativa}

% A seção de Justificativa, também pode ser chamada de Motivação. Esta seção deve deixar claro a importância do trabalho e do assunto abordado, e explorar o que motivou a realização do projeto. Esta seção não deve ser muito longa e se resume em alguns parágrafos.

    
