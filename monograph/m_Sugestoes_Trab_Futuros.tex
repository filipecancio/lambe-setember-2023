\chapter{Sugestões para Trabalhos Futuros}
    Como sugestão para trabalhos futuros sugere-se a avaliação da plataforma em ambiente real, após o seu uso e divulgação a fim de mensurar sua efetividade e identificar possíveis melhorias. Além disso, é importante considerar a melhoria na gestão dos animais cadastados na plataforma. Isso pode envolver a criação de meios para que todos processos relacionados às situações de alertas possam ser resolvidos dentro da plataforma e também com a implementação de um sistema de acompanhamento que permita obter dados sobre a destinação dos animais divulgados na rede social, como informações sobre adoções, tratamentos médicos, animais e tutores encontrados.
    % É fato que os trabalhos científicos nas áreas de ciência e de Engenharia não são concluídos completamente até a data da defesa.
    
    % Em discussão do estudante com o orientador, decide-se até onde avançar no trabalho, senão o trabalho nunca acaba, pois sempre há algo mais a fazer, algo mais a pesquisar, algo mais a aperfeiçoar, algo mais a acrescentar, algo mais a calibrar, algo mais a testar,\ldots

    % Nesse sentido evitamos a utilização do nome {\bf Conclusão}, e adotamos o nome {\bf Considerações Finais} ao fim do trabalho, e adotamos um capítulo para fazer comentários a respeito das {\bf Sugestões para Trabalhos Futuros}.

    % Neste capítulo o estudante comenta e dá sugestões para que o projeto possa ser continuado por outro estudante, e dá dicas do que faltou, o que pode ser melhorado, calibrado, testado, validado, etc\ldots