\clearpage\newpage\pagebreak%
\pagestyle{plain}

\Resumo 
Este trabalho descreve o desenvolvimento da plataforma web {\bf MeuPetAqui}, uma rede social para donos de animais de estimação e Organizações Não Gorvenamentais (\gls{ONGs}) envolvidas na proteção animal. A plataforma permite a criação de postagens, conexão entre usuários, promoção da adoção, busca por tutores de animais encontrados e auxílio no rastreamento de animais perdidos. Além disso, inclui ferramentas como o Registro Geral Animal (\gls{RGA}) para identificação do pet e de seu tutor, e o controle de vacinação, facilitando a gestão de vacinas e medicamentos recebidos pelo animal. A plataforma também atua como meio de divulgação de campanhas de arrecadação de recursos para tratamento de animais cuidados pelas ONGs.

A plataforma é utilizada por dois tipos de usuários: usuários comuns, que podem criar postagens, perfis para seus animais de estimação e emitir alertas relacionados à adoção, pets encontrados ou perdidos; e ONGs, que possuem as mesmas funcionalidades, além de alertas para animais que necessitam de cuidados veterinários. Os posts são exibidos na timeline principal, enquanto uma segunda timeline é dedicada exclusivamente aos alertas, destacando a importância da conscientização e do engajamento em causas do bem-estar animal.

O trabalho aborda a criação de um sistema que promove a conscientização e engajamento em questões relacionadas ao bem-estar animal. A plataforma MeuPetAqui é uma ferramenta para conectar simpatizantes da causa, proprietários de animais de estimação e ONGs, contribuindo para a redução do número de animais abandonados e promovendo relacionamentos saudáveis entre donos e seus animais de estimação.
%{\bf Título}
% máximo de 6 palavras chave separadas por vírgula
\begin{keywords}
plataforma web, MeuPetAqui, rede social, animais de estimação, rastreamento,ONGs
\end{keywords}