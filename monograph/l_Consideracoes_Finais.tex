\chapter{Considerações Finais}
O objetivo deste trabalho foi desenvolver uma plataforma web para auxiliar nas atividades recorrentes de ONGs e Protetores de animais em Vitória da Conquista, visando a redução do número de cães e gatos em situação de abandono. Além disso, buscou-se proporcionar um espaço de interação entre os usuários e formentar o bem-estar dos animais, promovendo ações em benefício deles.

A principal contribuição deste trabalho é a criação de uma rede social dedicada aos tutores de pets, ONGs e Protetores de animais. Essa plataforma permite a interação entre os usuários, possibilitando a criação de posts, perfis para os pets e alertas para divulgar situações em que os usuários necessitam de ajuda, como encontrar um pet perdido, realizar uma adoção, localizar o tutor de um pet ou também, ao se tratar de ONGs, buscar contribuições financeiras para o custeio de tratamento veterinário. Além destas funcionalidades, a plataforma também inclui o rastreamento de pets perdidos. Isso permite que os usuários registrem informações sobre animais desaparecidos e busquem ajuda da comunidade para localizá-los.

O sistema desenvolvido é uma aplicação web, com um layout intuitivo e de fácil adaptação pelos usuários. Através de elementos de design que seguem padrões familiares em serviços similares, proporcionamos uma experiência amigável e acessível.
\newpage
Com o objetivo de promover a continuidade do projeto e reforçar nosso compromisso em contribuir para a causa animal, os repositórios dos projetos desenvolvidos durante este trabalho estão disponibilizados nos links abaixo:


\begin{table}[htbp]
\centering
\renewcommand{\arraystretch}{1.5}
\begin{tabular}{|c|>{\centering\arraybackslash}m{6cm}|c|}
\hline
\textbf{Projeto} & \textbf{Link} & \textbf{QR Code} \\
\hline
Front-end & \href{https://github.com/EdiomarNogueira/front_mypet}{Repositório GitHub-Front-End} & \includegraphics[width=3.5cm]{arquivos/ImgLinks/qrcodeFront.png} \\
\hline
Back-end & \href{https://github.com/EdiomarNogueira/api_mypet}{Repositório GitHub-Back-End} & \includegraphics[width=3.5cm]{arquivos/ImgLinks/qrcodeBack.png} \\
\hline
\end{tabular}
\end{table}


Esses repositórios estão abertos ao público e são uma forma de compartilhar o conhecimento e permitir que outros estudantes, pesquisadores, desenvolvedores e entusiastas da causa animal possam se beneficiar do trabalho realizado. Encorajamos a exploração e, se necessário, adaptar ou estender esses projetos de acordo com as necessidades individuais.

Em resumo, este projeto representa a criação de uma plataforma web voltada para a interação e suporte às atividades de ONGs e Protetores de animais em Vitória da Conquista. Embora não tenha sido possível realizar testes para validar sua eficácia, acredita-se que a implementação dessa rede social possa contribuir significativamente para o trabalho de proteção animal e promover o cuidado e bem-estar dos pets na região. 
%     Em trabalhos científicos e de Engenharia, evitamos a utilização do nome {\bf Conclusão}, e adotamos o nome {\bf Considerações Finais} ao fim do trabalho, e adotamos um capítulo para fazer comentários a respeito das {\bf Sugestões para Trabalhos Futuros}.

%     Neste capítulo de {\bf Considerações Finais}, deve conter uma breve comentário a respeito das etapas de realização do trabalho, desafios e dificuldades encontradas.    
%     Este capítulo deve estar alicerçado nos {\bf resultados}, deve associá-los à confirmação (ou não) da(s) hipótese(s) ou pressuposto(s) se for o caso, e aos objetivos estabelecidos.

%     Incluir propostas e recomendações para implementação de resultados e novas pesquisas, dando fechamento ao trabalho.
    
%     Segundo a norma \cite{NBR14724_TrabalhosAcademicos} este capítulo trata-se de: Parte final do texto, na qual se apresentam conclusões correspondentes aos objetivos ou hipóteses. 

% %\newpage % comando que transfere o texto para a próxima página
% \section{Dicas importantes:}
    
% \begin{itemize}
%     \item Evite conclusões óbvias;
%     \item Evite se auto promover, exaltar ou idolatrar o seu trabalho;
%     \item As conclusões devem ser curtas (uma ou duas páginas) e baseadas no que foi escrito em seu trabalho;
%     \item Cite os resultados encontrados e os objetivos e metas alcançadas e não alcançadas;
%     \item É interessante que você recomende alguma contribuição para trabalhos futuros, que possa ajudar alguma pessoa a continuar ou repetir o seu trabalho;
%     \item Aqui não devem ser colocadas tabelas nem figuras, nem equações;
%     \item Aqui não deve ser colocada conclusões de outras pessoas, nem textos de outras pessoas, a não ser que sejam comparações de seus resultados com outros já sacramentados no Estado da Arte;
%     \item Evitar afirmações duras e genéricas, difíceis de provar na prática;
%     \item A conclusão deve possuir Clareza, Coesão e Coerência:
%     \begin{itemize}
%         \item {\bf Clareza:} bem explicado e de fácil entendimento;
%         \item {\bf Coesão:} deve reunir e ligar todas as partes do seu trabalho;
%         \item {\bf Coerência:} raciocínio sem contradições;
%     \end{itemize}
%     \item As Considerações Finais, juntamente com o resumo, são as seções mais importantes do texto do seu trabalho, e devem ser feitas cuidadosamente para valorizar todo o seu trabalho; 
% \end{itemize}